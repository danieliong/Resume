%%%%%%%%%%%%%%%%%
% This is an example CV created using altacv.cls (v1.1.3, 30 April 2017) written by
% LianTze Lim (liantze@gmail.com), based on the 
% Cv created by BusinessInsider at http://www.businessinsider.my/a-sample-resume-for-marissa-mayer-2016-7/?r=US&IR=T
% 
%% It may be distributed and/or modified under the
%% conditions of the LaTeX Project Public License, either version 1.3
%% of this license or (at your option) any later version.
%% The latest version of this license is in
%%    http://www.latex-project.org/lppl.txt
%% and version 1.3 or later is part of all distributions of LaTeX
%% version 2003/12/01 or later.
%%%%%%%%%%%%%%%%

%% If you want to use \orcid or the
%% academicons icons, add "academicons"
%% to the \documentclass options. 
%% Then compile with XeLaTeX or LuaLaTeX.
% \documentclass[10pt,a4paper,academicons]{altacv}

%% Use the "normalphoto" option if you want a normal photo instead of cropped to a circle
% \documentclass[10pt,a4paper,normalphoto]{altacv}

\documentclass[9pt,a4paper]{altacv}


%% AltaCV uses the fontawesome and academicon fonts
%% and packages. 
%% See texdoc.net/pkg/fontawecome and http://texdoc.net/pkg/academicons for full list of symbols.
%% When using the "academicons" option,
%% Compile with LuaLaTeX for best results. If you
%% want to use XeLaTeX, you may need to install
%% Academicons.ttf in your operating system's font %% folder.


% Change the page layout if you need to
\geometry{left=7mm,right=9cm,marginparwidth=6.8cm,marginparsep=.5cm,top=1cm,bottom=1cm}

% Change the font if you want to.

% If using pdflatex:
\usepackage[utf8]{inputenc}
\usepackage[T1]{fontenc}
\usepackage[default]{lato}

\usepackage{eurosym}
\usepackage{amstext} % for \text
\DeclareRobustCommand{\officialeuro}{%
  \ifmmode\expandafter\text\fi
  {\fontencoding{U}\fontfamily{eurosym}\selectfont e}}

% If using xelatex or lualatex:
% \setmainfont{Lato}


% Change the colours if you want to
\definecolor{VividPurple}{HTML}{100000}
\definecolor{SlateGrey}{HTML}{2E2E2E}
\definecolor{LightGrey}{HTML}{666666}
\colorlet{heading}{VividPurple}
\colorlet{accent}{VividPurple}
\colorlet{emphasis}{SlateGrey}
\colorlet{body}{LightGrey}

% Change the bullets for itemize and rating marker
% for \cvskill if you want to
\renewcommand{\itemmarker}{{\small\textbullet}}
\renewcommand{\ratingmarker}{\faCircle}


%% sample.bib contains your publications
\addbibresource{publications.bib}

% \usepackage{fontawesome5}

\begin{document}

\name{Daniel Iong}
\tagline{Ph.D Candidate in Statistics}
% Cropped to square from https://en.wikipedia.org/wiki/Marissa_Mayer#/media/File:Marissa_Mayer_May_2014_(cropped).jpg, CC-BY 2.0
% \photo{2.5cm}{Um_pic.jpg}
\personalinfo{%
  % Not all of these are required!
  % You can add your own with \printinfo{symbol}{detail}

  \vspace{3pt}
  \homepage{danieliong.me}
  \github{https://github.com/danieliong}
  \linkedin{https://www.linkedin.com/in/danieliong/}
  \vspace{pt}

  \location{Ann Arbor, MI} 
  % \mailaddress{635 Hidden Valley Club Drive Apt 211, Ann Arbor MI 48104}
  \email{daniong@umich.edu}
  \phone{510-816-8686}
  \printinfo{\faPassport}{US Citizen}
  
%   \orcid{orcid.org/0000-0000-0000-0000} % Obviously making this up too. If you want to use this field (and also other academicons symbols), add "academicons" option to \documentclass{altacv}
}

%% Make the header extend all the way to the right, if you want.
\begin{fullwidth}
\makecvheader
\end{fullwidth}

\setlength\marginparwidth{8cm}

% \cvsection[page1sidebar]{Education}

% \cvevent{Ph.D Statistics}{University of Michigan, Ann Arbor}{2017 - 2022 (expected)}{Ann Arbor, MI}

% \divider

% \cvevent{B.S. Statistics (high honors) \\ B.A. Economics (honors)}{University of California, Davis}{2013 - 2017}{Davis, CA}


%% Provide the file name containing the sidebar contents as an optional parameter to \cvsection.
%% You can always just use \marginpar{...} if you do
%% not need to align the top of the contents to any
%% \cvsection title in the "main" bar.
\cvsection[page1sidebar]{Research Experience}

\cvevent{Graduate Student Research Assistant}{University of Michigan, Ann Arbor, Department of Statistics}{Aug. 2018 - present}{}

\vspace{4pt}

\begin{project}{A Latent Mixture Model for Heterogeneous Causal Mechanisms in Mendelian Randomization}{\faChartLine}
  % \setlength{\parskip}{2mm}
  % \setlength\itemsep{2mm}
  % \def\labelitemi{\faChartLine}
  \item Developed novel probabilistic clustering method for causal inference
  in Epidemiology (Mendelian Randomization)
  \item Implemented Monte-Carlo EM algorithm in C++ and R to perform
    statistical inference. 
  \item Invited to present method and results to statisticians and medical
    researchers at international seminars 
  \item[\faGithub] Developed R package: \url{https://github.com/danieliong/MRPATH}
  \item[\faLink] Created website to showcase method: \url{danieliong.me/mr-path/}
\end{project}

\divider

\begin{project}{Machine learning methods for predicting geomagnetic indices (ongoing)}{\faChartLine}
  \item Applied machine learning and time series methods to forecast geomagnetic
  activity in collaboration with space weather researchers
  \item Wrote data pre-processing and analysis tools tailored for geomagnetic
  data using Scikit-learn, Tensorflow, Pytorch
  \vspace{2pt}
  \begin{itemize}[leftmargin=18pt]
    \item[\faGithub] Python module: \url{https://github.com/danieliong/GeoMagTS}
  \end{itemize}
  \item[\faGithub] Code for analysis: \url{https://github.com/danieliong/SYMH-Prediction}  
\end{project}

% \vspace{-8pt}
\vspace{7pt}

\cvevent{Undergraduate Research Assistant}{University of California, Davis, Department of Statistics}{Apr. 2016 - Apr. 2017}{}

\vspace{4pt}

\begin{project}{NSF-funded Research Project: Predicting Dynamics for functional data}{\faChartLine}
  \item Analyzed economic data using functional data analysis methods in R.
  \item Extended existing empirical dynamics model to include covariates to increase coefficient of determination.  
\end{project}

\divider

\begin{project}{Undergraduate Honors Thesis: Toward a spatial-temporal analysis of pesticide concentrations}{\faChartLine}
  \item Implemented EM algorithm in R to fit state-space model to pesticide
  concentrations data containing missing values. 
\end{project}

\divider 

\begin{project}{NSF-funded Research Project: Processing and analyzing data from the Human Connectome Project}{\faChartLine}
  \item Applied principal components analysis and canonical correlation analysis
  to study the relationships between behavioral and cortical measures in R
\end{project}

% \vspace{3pt}

\cvsection{Teaching Experience}

\cvevent{Graduate Student Instructor}{University of Michigan, Ann Arbor}{Aug. 2018 - Apr. 2020}{}

% \vspace{4pt}

\projectheader{STATS551: Bayesian Modeling and Computation (graduate)}{\faChalkboardTeacher}

\projectheader{STATS451: Bayesian Data Analysis (undergraduate)}{\faChalkboardTeacher}

\projectheader{STATS 250: Intro. to Statistics and Data Analysis (undergraduate)}{\faChalkboardTeacher}

\smallskip

\begin{itemize}
  \item Prepared lectures on advanced topics in Bayesian modeling.
  \item Created and graded homework assignments.
  \item Advised students on exclusive projects in applied Bayesian analysis.
  \item Taught two weekly labs on basic concepts in statistics.
  \item Held weekly office hours to answer questions about course material and homework. 
\end{itemize}

% \begin{project}{STATS 551: Bayesian Modeling and Computation (graduate)}{\faChalkboardTeacher}
%   \item Prepared several lectures on advanced topics in Bayesian modeling.
%   \item Created and graded homework assignments with two other PhD students.
%   \item Advised students on extensive projects in applied Bayesian analysis.
% \end{project}

% \divider

% \begin{project}{STATS 451: Bayesian Data Analysis (undergraduate)}{\faChalkboardTeacher}
%   \item Created and graded bi-weekly homework assignments.
%   \item Held weekly office hours to answer questions about homework and lectures.
%   \item Advised students on extensive projects in applied Bayesian analysis.
% \end{project}

% \divider

% \begin{project}{STATS 250: Intro. to Statistics and Data Analysis (undergraduate)}{\faChalkboardTeacher}
%   \item Taught two weekly lab sessions on basic concepts in statistics.
%   \item Held office hours and graded homeworks on a weekly basis.
% \end{project}


\clearpage


%% If the NEXT page doesn't start with a \cvsection but you'd
%% still like to add a sidebar, then use this command on THIS
%% page to add it. The optional argument lets you pull up the 
%% sidebar a bit so that it looks aligned with the top of the
%% main column.
% \addnextpagesidebar[-1ex]{page3sidebar}


\end{document}
